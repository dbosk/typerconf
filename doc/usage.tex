The configuration is a JSON structure.
We'll use the following for the coming examples.
\label{ConfigStructure}
\begin{minted}{JSON}
{
  "courses": {
    "datintro22": {
      "timesheet": {
        "url": "https://sheets.google..."
      },
      "schedule": {
        "url": "https://timeedit.net/..."
      }
    }
  }
}
\end{minted}
The format is actually irrelevant to anyone outside of this library, since it 
will never be accessed directly anyway.
But it will be used to illustrate the examples.

We can access values by dot-separated addresses.
For instance, we can use \texttt{courses.datintro22.schedule.url} to access the 
TimeEdit URL of the datintro22 course.

Let's have a look at some usage examples.

\subsubsection{A command-line application}

Say we have the program \texttt{nytid} that wants to use this config module and 
subcommand.
We want the CLI command to have the following form when used with 
\texttt{nytid}.
\begin{minted}{bash}
  nytid config courses.datintro22.schedule.url --set https://timeedit.net/...
\end{minted}
will set the configuration value at the path, whereas
\begin{minted}{bash}
  nytid config courses.datintro22.schedule.url
\end{minted}
will return it.

We can do this with both \texttt{typer} (and \texttt{click}) and \texttt{argparse}.
With \texttt{typer}:
\begin{minted}{python}
import typer
import typerconf as config

cli = typer.Typer()
config.add_config_cmd(cli)
\end{minted}

With \texttt{argparse}:
\begin{minted}{python}
import argparse
import typerconf as config
import sys

cli = argparse.ArgumentParser()
subparsers = cli.add_subparsers(dest="subcommand")
config.add_config_cmd(subparsers)

args = cli.parse_args()
if args.subcommand == "config":
  args.func(args)
\end{minted}

Internally, \texttt{nytid}'s different parts can access the config through the 
following API\@.
\begin{minted}{python}
import typerconf as config

url = config.get("courses.datintro22.schedule.url")
\end{minted}

\subsubsection{Without the CLI}

We can also use it without the CLI and application features.
Then it's the \texttt{typerconf.Config} class that is of interest.

Let's assume that we have the structure from above (\cref{ConfigStructure}) in 
the file \mintinline{bash}{~/.config/app.config}.
Consider the following code.
\begin{minted}[mathescape]{python}
defaults = {
  "courses": {
    "datintro22": {
      "root": "/afs/kth.se/..."
    }
  }
}

conf = Config(json_data=defaults, conf_file="~/.config/app.config")

print(f"datintro22 root directory = {conf.get('courses.datintro22.root')}")
print(f"datintro22 schedule = {conf.get('courses.datintro22.schedule')}")
\end{minted}
When we construct \mintinline{python}{conf} above, we merge the default config 
with the values set in the config file that was loaded.

We note that the construction of \mintinline{python}{conf} above, can be 
replaced by the equivalent
\begin{minted}[linenos=false]{python}
conf = Config(defaults)
conf.read_config("~/.config/app.config", writeback=True)
\end{minted}
The meaning of \mintinline{python}{writeback} is that whenever we change the 
config, it will automatically be written back to the file from which it was 
read.
Writeback is enabled by default when supplying the file to the constructor.
It's disabled by default for the method \mintinline{python}{conf.read_config}, 
but we can turn it on by passing the \mintinline{python}{writeback=True}.

We can change the config by using the \mintinline{python}{conf.set} method.
\begin{minted}[firstnumber=13]{python}
conf.set("courses.datintro22.root", "/home/dbosk/...")
\end{minted}
That would change the root that we got from the default config.
Since we enabled writeback above, this would automatically update the config 
file with the new values.

