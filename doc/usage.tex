The configuration is a JSON structure.
We'll use the following for the coming examples.
\label{ConfigStructure}
\begin{minted}{JSON}
{
  "courses": {
    "datintro22": {
      "timesheet": {
        "url": "https://sheets.google..."
      },
      "schedule": {
        "url": "https://timeedit.net/..."
      }
    }
  }
}
\end{minted}
The format is actually irrelevant to anyone outside of this library, since it 
will never be accessed directly anyway.
But it will be used to illustrate the examples.

We can access values by dot-separated addresses.
For instance, we can use \texttt{courses.datintro22.schedule.url} to access the 
TimeEdit URL of the datintro22 course.

Let's have a look at some usage examples.
Say we have the program \texttt{nytid} that wants to use this config module and 
subcommand.
\begin{minted}{python}
import typer
import typerconf as config

cli = typer.Typer()
# add some other subcommands
config.add_config_cmd(cli)
\end{minted}

We want the CLI command to have the following form when used with \texttt{nytid}.
\begin{minted}{bash}
  nytid config courses.datintro22.schedule.url --set https://timeedit.net/...
\end{minted}
will set the configuration value at the path, whereas
\begin{minted}{bash}
  nytid config courses.datintro22.schedule.url
\end{minted}
will return it.

Internally, \texttt{nytid}'s different parts can access the config through the 
following API.
\begin{minted}{python}
import typerconf as config

url = config.get("courses.datintro22.schedule.url")
\end{minted}

